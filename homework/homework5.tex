\documentclass[letter]{article}
\usepackage{amsmath}
\usepackage{amsfonts}
\usepackage{amssymb}
\usepackage{ifthen}
\usepackage{fancyhdr}
\usepackage{enumitem}
\usepackage[hidelinks]{hyperref}
\usepackage{tikz}

%%%
% Set up the margins to use a fairly large area of the page
%%%
\oddsidemargin=.2in
\evensidemargin=.2in
\textwidth=6in
\topmargin=-.4in
\textheight=9.0in
\parskip=.07in
\parindent=0in
\pagestyle{fancy}

%%%
% Set up the header
%%%
\newcommand{\setheader}[6]{
	\lhead{{\sc #1}\\{\sc #2} ({\small \it \today})}
	\rhead{
		{\bf #3} 
		\ifthenelse{\equal{#4}{}}{}{(#4)}\\
		{\bf #5} 
		\ifthenelse{\equal{#6}{}}{}{(#6)}%
	}
}

%%%
% Set up some shortcut commands
%%%
\newcommand{\R}{\mathbb{R}}
\renewcommand{\C}{\mathbb{C}}
\newcommand{\N}{\mathbb{N}}
\newcommand{\Z}{\mathbb{Z}}
\newcommand{\Proj}{\mathrm{proj}}
\newcommand{\Perp}{\mathrm{perp}}
\newcommand{\proj}{\mathrm{proj}}
\newcommand{\Span}{\mathrm{span}}
\newcommand{\Null}{\mathrm{null}}
\newcommand{\Rank}{\mathrm{rank}}
\newcommand{\mat}[1]{\begin{bmatrix}#1\end{bmatrix}}
\renewcommand{\d}{\mathrm{d}}

%%%
% This is where the body of the document goes
%%%
\begin{document}
	\setheader{Math 281-3}{Homework 5}{Due Thursday, May 5}{}{}{}

	\begin{enumerate}
		\item For each of the following statements, produce a counterexample to show that the statement is {\bf false}.
			\begin{enumerate}
				\item If $A$ and $B$ are square matrices, $AB=BA$.
				\item If $AB=\mat{1&1\\1&1}$, then $A$ and $B$ are $2\times 2$ matrices.
				\item If $AB=I$ then $BA=I$.
				\item If $A^2=0$, then $A=0$.
			\end{enumerate}
		\item Let $R=\mat{1&2&3\\4&5&6\\7&8&9}$.
			\begin{enumerate}
				\item Find all solutions to the matrix equation $R\mat{x_1\\x_2\\x_3}=\mat{2\\5\\8}$.
				\item Prove that the set $X=\{\vec x\in\R^3:R\vec x=\vec 0\}$ is a subspace.
				\item Prove that the set $Y=\{\vec y\in\R^3:\vec y=R\vec x\text{ for some }\vec x\in\R^3\}$ is a subspace.
			\end{enumerate}
		\item Suppose $E$ is a $4\times 3$ matrix with columns $\vec c_1,\vec c_2,\vec c_3$ and rows
			$\vec r_1,\vec r_2,\vec r_3,\vec r_4$.  Let $\vec v=\mat{2\\-1\\1}$.
			\begin{enumerate}
				\item Express $E\vec v$ as a linear combination of $\vec c_1,\vec c_2,\vec c_3$.
				\item Supposing $\vec r_1\cdot \vec v=1$, $\vec r_2\cdot \vec v=6$, $(\vec r_3+\vec r_4)\cdot \vec v=2$,
					and $(\vec r_3-\vec r_4)\cdot \vec v=-2$, compute $E\vec v$.
			\end{enumerate}

		\item Suppose that $\vec u$, $\vec v$, and $\vec w$ are vectors in $\R^2$ that are related
			by the following diagram.
		\begin{center}
			\begin{tikzpicture}[scale=.5]
				\draw[-] (-6, 0) -- node [below,
				very near end] {} (6, 0);
				\draw[-] (0, -2) -- node [right,
				very near start] {} (0, 5);
				\draw[->] (0, 0) -- node [below,
				very near end] {$4\vec{u}$}
				(3, -1);
				\draw[->] (0, 0) -- node [left,
				near end] {$2\vec{v}$} (2, 3);
				\draw[->] (0, 0) -- node [left,
				very near end] {$3\vec{w}$}
				(-1, 4);
				\draw[dotted] (3, -1) -- node
				[right, very near end] {} (2, 3);
				\draw[dotted] (-1,4) -- node
				[below, very near end] {} (2, 3);
			\end{tikzpicture}
		\end{center}
		Let $A=[\vec u|\vec v|\vec w]$ be the matrix with columns $\vec u$, $\vec v$, and $\vec w$.
		\begin{enumerate}
			\item What is the rank of $A$?
			\item Find all solutions to the equation $A\vec x=\vec 0$.
			\item Find a basis for the subspace $V=\{\vec x\in\R^3: A\vec x=\vec 0\}$.
		\end{enumerate}

		\item {\sc Linear transformations were here the whole time!} For each transformation, determine
			whether or not it is a linear transformation.  If it is a linear transformation, give
			its null space.
			\begin{enumerate}
				\item Let $V=\Span\{1,x,x^2\}$ where $1,x,x^2$ are (as usual) polynomials.
					Let $D:V\to V$ be differentiation.  (Use the limit definition of 
					the derivative. However, you can
					do limits like you did in Calc I; you don't have to do $\varepsilon$--$\delta$
					proofs).
				\item Let $\mathcal F=\{f:\R\to \R\}$ and let $D^s:\mathcal F\to \mathcal F$
					be left-translation by $s$.  That is, $D^sf(t)=f(t+s)$.
				\item Let $\mathcal F=\{f:\R\to \R\}$ and let $D^s:\mathcal F\to \mathcal F$
					be vertical-translation by $s$.  That is, $D^sf(t)=f(t)+s$.
				\item Let $\mathcal F=\{f:\R\to \R\}$ and let $D^s:\mathcal F\to \mathcal F$
					be horizontal-stretching by $s$.  That is, $D^sf(t)=f(t/s)$.
				\item Let $\mathcal F=\{f:\R\to \R\}$ and let $D^s:\mathcal F\to \mathcal F$
					be vertical-stretching by $s$.  That is, $D^sf(t)=sf(t)$.
				\item Let $\mathcal I=\{f:\R\to \R\mid f\text{ is increasing}\}$ and let $D:\mathcal I\to \mathcal I$
					be so that $D(f)=f^{-1}$ takes a function to its inverse.
				\item Let $\mathcal L=\{f:\R\to \R\mid f(x)=mx+b\text{ for some }m,b\in\R\}$
					be the set of all functions whose graphs are non-vertical lines and let $D:\mathcal L\to \R$
					be evaluation at the point $x=1$.  That is, $D(f) = f(1)$.
			\end{enumerate}

		\item {\sc Once upon a time we were doing vector calculus!}
			\begin{enumerate}
				\item Let $f:\R^2\to \R$ be defined by $f(x,y) = \cos(x)+\sin(y)$.  Let $D^f:\R^2\to\R$
					be such that $D^f(\vec v)$ is the directional derivative of $f$ in at the point $(0,0)$
					in the direction $\vec v$.  Show, using the definition of directional derivative, that
					$D^f$ is linear.
				\item Let $\mathcal F=\{\text{differentiable functions from $\R^2$ to $\R$}\}$ and let $\vec v\in \R^2$.
					Define $D^{\vec v}:\mathcal F\to \R$ to be the function such that $D^{\vec v}(h)$
					is the directional derivative of $h$ at the point $(0,0)$ in the direction $\vec v$.
					Show that $D^{\vec v}$ is a linear transformation.
				\item Let $D_{\vec x}:\mathcal F\times \R^2\to\R$ be so that $D_{\vec x}(h,\vec v)$ is the directional
					derivative of $h$ at the point $\vec x$ in the direction $\vec v$.
					Take a moment to appreciate that $D_{\vec x}$ is \emph{bilinear}.  That is, it is
					linear in both its first and second arguments.  Let $\vec f:\R^2\to\R^2$ be a 
					vector field.  Notice that $D_{\vec x}(h,\vec f(\vec x))$ is still a linear function of
					$h$.  The function $d_{\vec x,\vec f}(h)=D_{\vec x}(h, \vec f(\vec x))$ is called
					a \emph{derivation}.  Much of what we did in multi-variable calculus can be rephrased
					in terms of derivations.    Bilinearity of $D_{\vec x}$
					ensures that for a fixed $\vec x$, the set of all derivations is itself a vector space.
					This vector space is called the \emph{tangent space at $\vec x$}, and it is equivalent
					to the notion of tangent planes we developed first term.

					Think about the functions defined in the previous paragraph until their definitions
					make sense.  Then, contemplate whether a differential geometry course (where these
					ideas are studied in depth) would interest you.  You do not need to write anything for this part.
				\item In Math 281-1, homework 3, problem 5, we considered the function $f(x,y)=\frac{x^2y}{x^2+y^2}$
					if $(x,y)\neq(0,0)$ and $f(0,0)=0$.  Let $D^f:\R^2\to \R$ be so that $D^f(\vec v)$
					is the directional derivative
					of $f$ at the point $(0,0)$ in the direction $\vec v$.  Is $D^f$ linear?  Make a conjecture
					about the relationship between linearity of the directional derivative operator
					and differentiability of a function.
			\end{enumerate}

	\end{enumerate}


\end{document}
