\documentclass[letter]{article}
\usepackage{amsmath}
\usepackage{amsfonts}
\usepackage{amssymb}
\usepackage{ifthen}
\usepackage{fancyhdr}

%%%
% Set up the margins to use a fairly large area of the page
%%%
\oddsidemargin=.2in
\evensidemargin=.2in
\textwidth=6in
\topmargin=0in
\textheight=9.0in
\parskip=.07in
\parindent=0in
\pagestyle{fancy}

%%%
% Set up the header
%%%
\newcommand{\setheader}[6]{
	\lhead{{\sc #1}\\{\sc #2} ({\small \it \today})}
	\rhead{
		{\bf #3} 
		\ifthenelse{\equal{#4}{}}{}{(#4)}\\
		{\bf #5} 
		\ifthenelse{\equal{#6}{}}{}{(#6)}%
	}
}

%%%
% Set up some shortcut commands
%%%
\newcommand{\R}{\mathbb{R}}
\newcommand{\N}{\mathbb{N}}
\newcommand{\Z}{\mathbb{Z}}
\newcommand{\Proj}{\mathrm{proj}}
\newcommand{\Perp}{\mathrm{perp}}
\newcommand{\proj}{\mathrm{proj}}
\newcommand{\Span}{\mathrm{span}}
\newcommand{\Null}{\mathrm{null}}
\newcommand{\Rank}{\mathrm{rank}}
\newcommand{\mat}[1]{\begin{bmatrix}#1\end{bmatrix}}

%%%
% This is where the body of the document goes
%%%
\begin{document}
	\setheader{Math 211 (A01)}{Sample Homework}{Jason Siefken}{V00123456}{}{}
	\begin{enumerate}
		\item Show that the standard basis in $\R^3$ is linearly independent.
		\begin{quote}
			First recall that a set of vectors 
			$\{\vec v_1,\vec v_2,\ldots,\vec v_n\}$ is \emph{linearly independent}
			if the only way to satisfy the equation
			\[
				\vec 0 = \alpha_1\vec v_1+\alpha_2\vec v_2+\cdots+\alpha_n\vec v_n
			\]
			is when $\alpha_1=\alpha_2=\cdots=\alpha_n=0$.

			The \emph{standard basis} for $\R^3$ consists of the three vectors
			\[
				\vec e_1=\mat{1\\0\\0}\qquad
				\vec e_2=\mat{0\\1\\0}\qquad
				\vec e_3=\mat{0\\0\\1}.
			\]

			Considering the arbitrary linear combination $\vec w=\alpha_1\vec e_1
			+\alpha_2\vec e_2+\alpha_3\vec e_3$, we see that
			\[
				\vec w = \alpha_1\mat{1\\0\\0}+\alpha_2\mat{0\\1\\0}+\alpha_3\mat{0\\0\\1}
				=\mat{\alpha_1\\\alpha_2\\\alpha_3}.
			\]
			Thus, if $\vec w=\vec 0$, we must have that $\alpha_1=\alpha_2=\alpha_3=0$,
			and so $\{\vec e_1,\vec e_2,\vec e_3\}$ is linearly independent.
		\end{quote}

		\item Find the angle between the vectors $\vec v=\mat{1\\2}$
		and $\vec w=\mat{3\\4}$.
		\begin{quote}
			We can find the angle between $\vec v$ and $\vec w$ using 
			the dot product formula.

			The \emph{dot product} of two vectors $\vec a=\mat{a_1\\ a_2\\\vdots\\ a_n}$
			and $\vec b=\mat{b_1\\ b_2\\\vdots\\ b_n}$ is defined as
			\[
				\vec a\cdot \vec b = a_1b_1+a_2b_2+\cdots+a_nb_n.
			\]
			The dot product of two vectors can also be computed by the formula
			\[
				\vec a\cdot \vec b = \|\vec a\|\|\vec b\|\cos\theta,
			\]
			where $\theta$ is the angle between $\vec a$ and $\vec b$.

			In order to use these formulae with $\vec v$ and $\vec w$, we
			must first compute some quantities. 
			\[
				\|\vec v\| = \sqrt{1^2+2^2}=\sqrt{5}\qquad
				\|\vec w\| = \sqrt{3^2+4^2}=5.
			\]
			Now, our dot product formulae gives
			\[
				\vec v\cdot\vec w = 1(3)+2(4) = 11=5\sqrt{5}\cos\theta,
			\]
			and so $\theta=\arccos\left(\frac{11}{5\sqrt{5}}\right)\approx 0.1799$ radians.
		\end{quote}

		\item For vectors $\vec u,\vec v,\vec w\in\R^n$, is it possible for
		$\Span\{\vec u,\vec v,\vec w\}=\Span\{\vec u,\vec v\}$?  Explain when it is
		and when it isn't.
		\begin{quote}
			The \emph{span} of a set of vectors is the set of all
			linear combinations of those vectors.

			Let $\vec u,\vec v,\vec w\in\R^n$ be vectors and suppose
			$\vec w$ is a \emph{linear combination} of $\vec u$ and $\vec v$.
			This means there are scalars $\alpha,\beta$ so that
			\[
				\vec w=\alpha\vec u+\beta\vec v.
			\]
			Since $\vec w$ is a linear combination of $\vec u$ and $\vec v$
			and $\Span\{\vec u,\vec v\}$ is the set of \emph{all}
			linear combinations of $\vec u$ and $\vec v$, we conclude that
			$\vec w\in\Span\{\vec u,\vec v\}$, and so adding $\vec w$
			to $\{\vec u,\vec v\}$ cannot enlarge its span.  Therefore,
			\[
				\Span\{\vec u,\vec v\}=\Span\{\vec u,\vec v,\vec w\}.
			\]

			On the other hand, if $\vec u$, $\vec v$, and $\vec w$ were
			linearly independent, none could be written as a linear combinations
			of the others.  Thus we would have $\vec w\notin\Span\{\vec u,\vec v\}$
			and so
			\[
				\Span\{\vec u,\vec v\}\subsetneq\Span\{\vec u,\vec v,\vec w\}.
			\]
		\end{quote}
	\end{enumerate}
\end{document}
