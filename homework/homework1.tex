\documentclass[letter]{article}
\usepackage{amsmath}
\usepackage{amsfonts}
\usepackage{amssymb}
\usepackage{ifthen}
\usepackage{fancyhdr}
\usepackage{enumitem}

%%%
% Set up the margins to use a fairly large area of the page
%%%
\oddsidemargin=.2in
\evensidemargin=.2in
\textwidth=6in
\topmargin=0in
\textheight=9.0in
\parskip=.07in
\parindent=0in
\pagestyle{fancy}

%%%
% Set up the header
%%%
\newcommand{\setheader}[6]{
	\lhead{{\sc #1}\\{\sc #2} ({\small \it \today})}
	\rhead{
		{\bf #3} 
		\ifthenelse{\equal{#4}{}}{}{(#4)}\\
		{\bf #5} 
		\ifthenelse{\equal{#6}{}}{}{(#6)}%
	}
}

%%%
% Set up some shortcut commands
%%%
\newcommand{\R}{\mathbb{R}}
\newcommand{\C}{\mathbb{C}}
\newcommand{\N}{\mathbb{N}}
\newcommand{\Z}{\mathbb{Z}}
\newcommand{\Proj}{\mathrm{proj}}
\newcommand{\Perp}{\mathrm{perp}}
\newcommand{\proj}{\mathrm{proj}}
\newcommand{\Span}{\mathrm{span}}
\newcommand{\Null}{\mathrm{null}}
\newcommand{\Rank}{\mathrm{rank}}
\newcommand{\mat}[1]{\begin{bmatrix}#1\end{bmatrix}}
\renewcommand{\d}{\mathrm{d}}

%%%
% This is where the body of the document goes
%%%
\begin{document}
	\setheader{Math 281-3}{Homework 1}{Due Thursday, April 7}{}{}{}
	\begin{enumerate}
		\item \begin{enumerate}
				\item Find all values of $x,y$ that satisfy the following relationships:
					\begin{align*}
						x+y&=7\\
						2x-3y&=13.
					\end{align*}
					What can you say about $\Span\left\{\mat{1\\2},\mat{1\\-3}\right\}$ and the 
					vector $\mat{7\\13}$?
				\item Find values of $x,y,z$ that satisfy the following relationships (your answer
					may involve ugly fractions):
					\begin{align*}
						x+2y+8z&=1\\
						4x+5y+8z&=2.
					\end{align*}
					What can you say about $\Span\left\{\mat{1\\4},\mat{2\\5},\mat{8\\8}\right\}$ and the 
					vector $\mat{1\\2}$?
				\item Let $\vec w=\mat{5\\-12}$, $\vec u=\mat{1\\1}$, and $\vec v=\mat{-1\\0}$.  
				Express $\vec w$ as a linear combination of $\vec u$ and $\vec v$.

			\end{enumerate}
		\item \begin{enumerate}
			\item Let 
				\[
					S=\Span\left\{\mat{1\\1\\1},\mat{1\\1\\0},\mat{0\\0\\1}\right\}.
				\]
				Is $S$ a point, line, plane, or all of $\R^3$?  Explain.
			\item Let $\vec u = \mat{3\\4\\1}$, $\vec v=\mat{4\\-4\\-4}$, and $\vec w=\mat{14\\0\\d}$.
				\begin{enumerate}
					\item For what value(s) of $d$ is $\Span\{\vec u,\vec v,\vec w\}$ a plane?
					\item Is there a value of $d$ so $\Span\{\vec u,\vec v,\vec w\}$ a line?  Explain.
				\end{enumerate}

			\end{enumerate}
		
		\item Let $\vec x=\mat{1\\1}$ and $\vec y=\mat{1\\-1}$.  Prove that $\{\vec x,\vec y\}$ is a 
			basis for $\R^2$.

		\item Let 
			\begin{align*}
				U&=\Span\left\{\mat{1\\2\\3},\mat{3\\2\\1},\mat{4\\4\\4}\right\},\\
				V&=\left\{\vec x\in\R^3:\vec x\cdot\mat{1\\1\\1}=0\right\},\\
				W&=U\cup V.
			\end{align*}
			For each subset $U$, $V$, $W$ of $\R^3$, show whether it is a subspace or not.  If it is a subspace,
			classify it as a point, line, plane, or all of $\R^3$.  Further, if it is a subspace, give a basis
			for it.
		\item Notice that no definitions we've used so far have referred to coordinates, just vectors
			and scalars (we haven't even demanded our scalars be $\R$).  This is on purpose
			because any collection of objects that can be added and scalar multiplied can be 
			treated as a ``vector.''
			
			Let $\vec u=1$, $\vec v = x$, and $\vec w = x^2$ be polynomials and let $Q=\Span\{\vec u,\vec v,\vec w\}$.
			
			\begin{enumerate}
				\item Is $4x+3\in Q$? What about $x^3-3x+2$?
				\item Describe $\Span\{\vec u,\vec v,\vec w\}$ and $\Span\{\vec v,\vec w\}$.
				\item Consider the differential equation
					\[
						y''-4y=0.
					\]
					Express all solutions as a span.
				\item Is the set $X=\{e^{it}, e^{-it}, \sin t, \cos t\}$ linearly independent if we allow
					complex scalars?  (You may reference homework from last term to answer this question.)
					Explain.
				\item Let $S=\Span\{\sin t,\cos t\}$ (we're back to real scalars now).  Is $\sin(t+\pi/6)\in S$?  Show that $S=\{\alpha\sin(t+\beta):\alpha,\beta\in \R\}$.
				\item Define the map $P:\R^2\to\R^2$ in the following way.  Given $(\alpha,\beta)\in \R^2$ with 
					$(\alpha,\beta)\neq \vec 0$, find the 
					unique $\alpha'\in [0,\infty)$
					and $\beta'\in[0,2\pi)$ such that $\alpha\sin t+\beta\cos t = \alpha'\sin(t+\beta')$.
						Define $P(\alpha,\beta)=(\alpha',\beta')$ and $P(0,0)=(0,0)$.

					Find a formula for $P$.  Is $P$ invertible?  If so, find a formula for its inverse?  
					Do these formulas look familiar?
			\end{enumerate}
	\end{enumerate}

\end{document}
