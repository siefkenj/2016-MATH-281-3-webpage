\documentclass[letter]{article}
\usepackage{amsmath}
\usepackage{amsfonts}
\usepackage{amssymb}
\usepackage{ifthen}
\usepackage{fancyhdr}
\usepackage{enumitem}
\usepackage[hidelinks]{hyperref}
\usepackage{tikz}

%%%
% Set up the margins to use a fairly large area of the page
%%%
\oddsidemargin=.2in
\evensidemargin=.2in
\textwidth=6in
\topmargin=-.4in
\textheight=9.0in
\parskip=.07in
\parindent=0in
\pagestyle{fancy}

%%%
% Set up the header
%%%
\newcommand{\setheader}[6]{
	\lhead{{\sc #1}\\{\sc #2}}
	\rhead{
		{\bf #3} 
		\ifthenelse{\equal{#4}{}}{}{(#4)}\\
		{\bf #5} 
		\ifthenelse{\equal{#6}{}}{}{(#6)}%
	}
}

%%%
% Set up some shortcut commands
%%%
\newcommand{\R}{\mathbb{R}}
\renewcommand{\C}{\mathbb{C}}
\newcommand{\N}{\mathbb{N}}
\newcommand{\Z}{\mathbb{Z}}
\newcommand{\Proj}{\mathrm{proj}}
\newcommand{\Perp}{\mathrm{perp}}
\newcommand{\proj}{\mathrm{proj}}
\newcommand{\Span}{\mathrm{span}}
\newcommand{\Null}{\mathrm{null}}
\newcommand{\Rank}{\mathrm{rank}}
\newcommand{\mat}[1]{\begin{bmatrix}#1\end{bmatrix}}
\renewcommand{\d}{\mathrm{d}}

%%%
% This is where the body of the document goes
%%%
\begin{document}
\setheader{Math 281-3}{Homework 6 (Typed)}{Due Tuesday, May 10}{}{}{}

	\begin{enumerate}
		\item Let $T:\R^n\to\R^m$ be a linear transformation.
		\begin{enumerate}
			\item Show that the null space of $T$ is a subspace of $\R^n$.
			\item Show that the range of $T$ is a subspace of $\R^m$.
		\end{enumerate}
		
		\item 
		\begin{enumerate}
			\item For a $4\times 3$ matrix $M$, must the column space of $M$ be identical to
				the column space of $\mathrm{rref}(M)$?
			\item For a $3\times 3$ matrix $N$ with $\Rank(N)=3$, must the column space of $N$
				be identical to the column space of $\mathrm{rref}(N)$?  Can the assumption
				that $\Rank(N)=3$ be dropped?
		\end{enumerate}
		
		\item For a linear transformation $L:\R^3\to\R^3$, we have the following information:
			\[
				L\mat{1\\1\\0}=\mat{2\\1\\1}\qquad
				L\mat{-1\\1\\0}=\mat{-3\\3\\3}\qquad
				L\mat{0\\0\\1}=\mat{0\\0\\0}.
			\]
		\begin{enumerate}
			\item Write down a matrix for $L$.
			\item Describe the range of $L$ as a point, line, plane, or hyperplane and
				give a basis for the range of $L$.
			\item Describe the null space of $L$ as a point, line, plane, or hyperplane and
				give a basis for the null space of $L$.
		\end{enumerate}

	\end{enumerate}


\end{document}
