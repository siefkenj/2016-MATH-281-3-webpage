\documentclass[letter]{article}
\usepackage{amsmath}
\usepackage{amsfonts}
\usepackage{amssymb}
\usepackage{ifthen}
\usepackage{fancyhdr}
\usepackage[usenames,dvipsnames,svgnames,table]{xcolor}
\usepackage{tikz}

%%%
% Set up the margins to use a fairly large area of the page
%%%
\oddsidemargin=.2in
\evensidemargin=.2in
\textwidth=6in
\topmargin=0in
\textheight=9.0in
\parskip=.07in
\parindent=0in
\pagestyle{fancy}

\expandafter\def\expandafter\quote\expandafter{\quote\sf\color{DarkGreen}}

%%%
% Set up the header
%%%
\newcommand{\setheader}[6]{
	\lhead{{\sc #1}\\{\sc #2} %({\small \it \today})
	}
	\rhead{
		{\bf #3} 
		\ifthenelse{\equal{#4}{}}{}{(#4)}\\
		{\bf #5} 
		\ifthenelse{\equal{#6}{}}{}{(#6)}%
	}
}

%%%
% Set up some shortcut commands
%%%
\newcommand{\R}{\mathbb{R}}
\newcommand{\N}{\mathbb{N}}
\newcommand{\Z}{\mathbb{Z}}
\newcommand{\Proj}{\mathrm{proj}}
\newcommand{\Perp}{\mathrm{perp}}
\newcommand{\Span}{\mathrm{span}}
\newcommand{\Null}{\mathrm{null}}
\newcommand{\Rank}{\mathrm{rank}}
\newcommand{\mat}[1]{\begin{bmatrix}#1\end{bmatrix}}

%%%
% This is where the body of the document goes
%%%
\begin{document}
\setheader{Math 211 (A01)}{Homework 2 (Typed)}{Due Tuesday, April 12}{}{}{}

	\begin{enumerate}
		\item Let $\vec u=\mat{1\\2\\3}$, $\vec v=\mat{4\\5\\6}$, and $\vec w=\mat{7\\8\\9}$. Explain whether
			the set $A=\{\vec u,\vec v,\vec w\}$ is a basis for $\R^3$.  
			Make sure to include all relevant definitions.

		\item Fix $\vec u,\vec v\in \R^n$.  Show that $\Span(\Span\{\vec u,\vec v\})=\Span\{\vec u,\vec v\}$.
			Make sure to include all relevant definitions.
		
		\item The worksheets define $\Proj_{\vec v}\vec u$ as the vector in the direction $\vec v$ such that
			$\vec u-\Proj_{\vec v}\vec u$ is orthogonal to $\vec v$.  Call this definition (a).  Your textbook
			defined $\Proj_{\vec v}\vec u$ as the vector $\frac{\vec u\cdot \vec v}{\vec v\cdot\vec v}\vec v$.
			Call this definition (b).  Show that definitions (a) and (b) are equivalent by showing 
			that that the vector arising from definition (b) must be the same as the vector
			arising from definition (a).
			In your answer, elaborate on definition (a) by including
			the definition of \emph{vector in the direction of $\vec v$} and \emph{orthogonal}. (Hint: in the past,
			a major stumbling block for writeups has been choosing a notation that clearly distinguishes 
			vectors from definition (a) and vectors from definition (b).)


	\end{enumerate}
\end{document}
