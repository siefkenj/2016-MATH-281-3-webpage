\documentclass[letter]{article}
\usepackage{amsmath}
\usepackage{amsfonts}
\usepackage{amssymb}
\usepackage{ifthen}
\usepackage{fancyhdr}
\usepackage{enumitem}
\usepackage[hidelinks]{hyperref}

%%%
% Set up the margins to use a fairly large area of the page
%%%
\oddsidemargin=.2in
\evensidemargin=.2in
\textwidth=6in
\topmargin=-.4in
\textheight=9.0in
\parskip=.07in
\parindent=0in
\pagestyle{fancy}

%%%
% Set up the header
%%%
\newcommand{\setheader}[6]{
	\lhead{{\sc #1}\\{\sc #2} ({\small \it \today})}
	\rhead{
		{\bf #3} 
		\ifthenelse{\equal{#4}{}}{}{(#4)}\\
		{\bf #5} 
		\ifthenelse{\equal{#6}{}}{}{(#6)}%
	}
}

%%%
% Set up some shortcut commands
%%%
\newcommand{\R}{\mathbb{R}}
\renewcommand{\C}{\mathbb{C}}
\newcommand{\N}{\mathbb{N}}
\newcommand{\Z}{\mathbb{Z}}
\newcommand{\Proj}{\mathrm{proj}}
\newcommand{\Perp}{\mathrm{perp}}
\newcommand{\proj}{\mathrm{proj}}
\newcommand{\Span}{\mathrm{span}}
\newcommand{\Null}{\mathrm{null}}
\newcommand{\Rank}{\mathrm{rank}}
\newcommand{\mat}[1]{\begin{bmatrix}#1\end{bmatrix}}
\renewcommand{\d}{\mathrm{d}}

%%%
% This is where the body of the document goes
%%%
\begin{document}
	\setheader{Math 281-3}{Homework 4}{Due Thursday, April 28}{}{}{}


	For this homework, you will be using \emph{Matlab} (or the free and open-source version of
	\emph{Matlab} called \emph{Octave}).  Include a clearly-labeled printout
	of the work you did in Matlab for each problem.
	
	Before you start, read through \\
	\centerline{\url{http://web.gps.caltech.edu/classes/ge11d/doc/matlab_Resource_Seminar.pdf}}\\
	for a short Matlab introduction.
	For additional documentation, see the Matlab website \\\centerline{\url{http://www.mathworks.com/help/matlab/index.html}}

	\textbf{Quick Tips:}

	\vspace{-1em}
	\begin{itemize}
		\item You can supress Matlab output by adding a `\texttt{;}'
		to the end of your Matlab command (this is useful for
		Matlab commands executed in loops).

		\item You can save plots/figures from the figure's dialog
		by selecting `Save as' and then choosing \texttt{png}
		or \texttt{jpg} in the file-types dropdown.

		\item You can include Matlab code in \LaTeX\ by copying
		and pasting it between \texttt{\textbackslash
		begin\{verbatim\}} \ldots\texttt{\textbackslash
		end\{verbatim\}}.
	\end{itemize}

	\begin{enumerate}
		\item Find the general solution to the system of equations corresponding to each augmented matrix.
			\begin{enumerate}
				\item \[\left[\begin{array}{ccccc|c}
				       1&    11&   -22&     0&     3&1 \\
				    5244&  1542& -3084&   576& 16308&2\\
				     108&  1188& -2376&     0&   324&3\\
				     264&   465&  -930&    25&   817&4\\
				    2166&  3729& -7458&   206&  6704&5
					\end{array}\right]\]
				\item \[\left[\begin{array}{ccccc|c}
						    7&  3&  2&  3&  7&  1 \\
						    9&  3&  2&  1& 10&  2\\
						    7&  5&  2&  2&  8&  3\\
						    4& 10&  3&  1& 10&  4\\
						    4&  6&  1&  3&  9&  5
				    \end{array}\right]
					\]
			\end{enumerate}
		
		\item Although the \texttt{rref} command will row-reduce for you, Matlab makes it easy
			to row reduce ``by hand.''  If \texttt{A} is a matrix, the row operation 
			$R_2\mapsto R_2-6R_1$ can be performed with the command \texttt{A(2,:)=A(2,:)-6*A(1,:)}
			and other row operations can be performed similarly.

			\begin{enumerate}
				\item Consider the $3\times 3$ matrix $A=\mat{1&4&0\\1&2&0\\0&0&0}$ and $B=\mat{1&0&0\\0&1&0\\0&0&1}$.
					You only need to perform two row operations to fully reduce $A$.  Call these two
					operations $R^{(1)}$ and $R^{(2)}$. Use Matlab to perform those
					row operations on $A$ and also perform the same row operations on $B$ (effectively, un-row
					reducing $B$).
				\item Let $B_1=R^{(1)}(B)$ and $B_2=R^{(2)}(B)$ be the matrices obtained from $B$ by applying
					the two row operations in isolation (in Matlab, you can name variables
					\texttt{B1} and \texttt{B2} if you like).  What happens when you perform the matrix product
					$B_1A$?  How about $B_2A$?  And $B_2B_1A$?  Finally compute the product $BA$, and explain
					why you get the results you do.
				\item Let $X=\mat{2&1&3\\2&1&3\\3&2&4}$.  Find a matrix $Y$ so that $YX=\mathrm{rref}(X)$.
			\end{enumerate}
		\item A loop in Matlab/Octave is written with the 
	\begin{verbatim}
for i=1:n,
    [loop content]
end
	\end{verbatim}
	syntax.

	For example, if we want the variable $x$ to be an accumulated sum of squares from $1^2$ to $15^2$ 
	(that is $x=1^2+2^2+\cdots+15^2$), we could write
	\begin{verbatim}
x=0;
for i=1:15,
    x = x + i^2;
end
	\end{verbatim}
	After executing this code $x$ will be the value $1240$.  The code works as follows: first we assign the value
	$0$ to $x$.  Then we enter the loop.  In the first iteration, when $i$ is 1, the new value of $x$ will be the current value,
	0, plus $1^2$.  The next time through the loop, $i$ is 2 and so we assign $x$ to be the current value, $1^2$, plus
	$2^2$, etc.. Once we have looped through with the value of $i$ being 15, we stop.

	If statements are written with the 
	\begin{verbatim}
if [var]==[val],
    [if content]
end
	\end{verbatim}
	syntax (note the double equals). 

	For example, if we wanted $x$ to be the sum of the squares of only the even numbers between 1 and 15, we might write:
	\begin{verbatim}
x=0;
for i=1:15,
    if round(i/2) == i/2,
        x = x + i^2;
    end
end
	\end{verbatim}

	Here, {\tt round(i/2)==i/2} is true when {\tt i/2} has no decimal places and so it is even.  Thus, when that happens
	(and only when that happens) we add the value $i^2$ to $x$.


	{\bf Question}:
	\begin{enumerate}
		\item The command \texttt{rand(3,1)} will generate a random $3\times 1$ column vector.  Find out what percentage
		of random triples of vectors $\{\vec a,\vec b,\vec c\}$ are linearly independent.  Sample at least 1000
		random triples to gather a statistic.

		Consider your result and come up with an explanation of why you got the percentage you did.

		{\bf Tips}: Build a loop to test this 1000 times, but start out
		with a much smaller loop;
		use `\texttt{;}' inside your loop so you don't get a bunch of unhelpful information printed to the screen;
		and use Matlab's built-in commands to help you like \texttt{rref} to row reduce, \texttt{rank} to find the
		rank, or \texttt{rand(3,3)} to make a $3\times 3$ matrix of random numbers.

		\item The command \texttt{round(rand(3,1))} will generate a random $3\times 1$ column vector of just zeros
			and ones.  Find out what percentage
			of random triples  $\{\vec a,\vec b,\vec c\}$ of zero-one vectors are linearly independent.  Sample at least 1000
			random triples to gather a statistic.

		Explain your result.  Why is it different or the same as the previous part?  (Hint: it may be useful
		to think of random zero-one vectors as lying on the vertices of the unit cube.)

	\end{enumerate}

	\item {\sc The power of the right definition}.  Our definition of projection was
		slightly complicated and hard to compute with, especially compared with the textbook's.
		However, the benefit of our definition is that it works, almost unchanged, in many more contexts.
		In this problem we will be projecting onto subspaces.

		{\bf Definition}  If $\vec v$ is a vector and $U$ is a subspace, \emph{the projection of 
		$\vec v$ onto $U$} is the vector $\vec w\in U$ such that $\vec v-\vec w$ is orthogonal to
		every vector in $U$.

		For the remainder of this problem, let 
		\[
			\vec a'=\mat{1\\1\\2}\qquad \vec b'=\mat{1\\1\\1}\qquad \vec c\,' = \mat{-1\\6\\1}
		\]
		and let $\vec a=\vec a'/\|\vec a'\|$, $\vec b=\vec b'/\|\vec b'\|$, and $\vec c=\vec c'/\|\vec c'\|$.

		\begin{enumerate}
			\item Let's make life a little easier first.  Show that if $\mathcal B$ is a basis
				for a subspace $U$, then $\vec w$ is orthogonal to every vector in $U$ if
				and only if $\vec w$ is orthogonal to every vector in $\mathcal B$.
			\item Let $U=\Span\{\vec a,\vec b\}$.  Analytically find the exact vector
				$\vec d=\proj_U\vec c$. (Hint, $U=\Span\{\vec a,\vec b\}=\Span\{\vec a',\vec b'\}$,
				so you can avoid some square roots!)
			\item Imagine you're an engineer who forgot how to do math but still wanted to
				find this projection.  ``Ah ha,'' you think, ``I could use Matlab to
				search for this vector!''

				In Matlab, `\texttt{'}' transposes a matrix.  So, if you want
				to compute $\vec a\cdot \vec b$, and $\vec a$ and $\vec b$ are column
				vectors, you could do \texttt{a'*b} in Matlab.

				You decide to try an brute force the problem.  Using the command \texttt{r = randn(1)*a + randn(1)*b}
				you can generate a random vector in $U$ (\texttt{randn(1)} produces a normally distributed
				random number, so it is equally likely to be positive or negative, unlike \texttt{rand} which always
				produces a number in $[0,1]$).  You decide, since computers are fast, you will generate
				1000 random vectors in $U$ and see which one most closely fits what the projection should fit.

				Since it is unlikely any of your random vectors will be \emph{exactly} the projection,
				you will need a \emph{score} function.  That is, you will need a function $s:\R^3\to[0,\infty)$
				such that $s(\vec v)=0$ if $\vec v=\vec d=\Proj_U\vec c$ and $s(\vec v) < s(\vec w)$ if
				$\vec v$ is ``closer to'' the correct projection than $\vec w$.  Your score function
				doesn't need to be complicated, but it shouldn't involve $\vec d$.  
				(If it could involve $\vec d$, a wonderful
				score function would be $s(\vec v) = \|\vec d-\vec v\|$.)

				With score function in hand, your guess for $\Proj_U\vec c$ is the random vector with the lowest score.

				Write Matlab code to execute this procedure.  If your estimate for $\Proj_U\vec c$ is $\vec p$,
				call $e=\|\vec p-\vec d\|$ the \emph{error} in your estimate.  Repeat the process of picking
				the best estimate from 1000 random vectors at least 100 times and report the \textbf{average}
				error.
			\item In the procedure outlined above, we could use any basis for $U$ to generate random 
				vectors.  Find a unit vector $\vec B$ that is orthogonal to $\vec a$
				such that $U=\Span\{\vec a,\vec B\}$ (Hint, think
				geometrically and use projections).  Repeat the guessing procedure above using
				\texttt{r = randn(1)*a + randn(1)*B} to generate your random vectors.  What is the average error
				using this new basis?  Why is it smaller?  Explain.
			\item (Optional) We can dramatically improve the our guessing from above by using an \emph{adaptive algorithm}.
				That is, instead of using \texttt{r = randn(1)*a + randn(1)*b} to generate a new
				random vector each time, we could use our previous best guess and see if we improve by perturbing
				it in a random direction.  Thus, our random vectors might look like \\
				\centerline{\texttt{r = bestGuessSoFar + (randn(1)*a + randn(1)*b)/10}}
				where we divide by 10 so that we don't guess something too far from our previous good guess 
				(a more advanced adaptive algorithm would use the score function and adjust the factor of 10
				based on that).  How much improvement do you see with an adaptive algorithm?
		\end{enumerate}

	\end{enumerate}

\end{document}
